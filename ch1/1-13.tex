\documentclass[12pt]{article}
\usepackage{amsmath}
\usepackage{amssymb}
\usepackage{geometry}

\setlength{\parskip}{1em}
\geometry{
	a4paper,
	total={170mm,257mm},
	left=20mm,
	top=20mm,
}

\title{SICP exercise 1.12.}
\author{}
\date{}

\begin{document}
\maketitle
\vspace{-2cm}
Below is presented an inductive proof of the Binet's formula for the $n^{\text{th}}$ Fibonacci number $F_n$.
For $n=1$, the formula reads $((1+\sqrt{5})/2 - (1-\sqrt{5})/2) / \sqrt{5} = 1 = F_1$. If the
equation is true for $n$, then
\begin{align*}
  F_{n+1} &= F_n + F_{n-1} \\
          &= \frac{(\frac{1+\sqrt{5}}{2})^n - (\frac{1-\sqrt{5}}{2})^n}{\sqrt{5}} + \frac{(\frac{1+\sqrt{5}}{2})^{n-1} - (\frac{1-\sqrt{5}}{2})^{n-1}}{\sqrt{5}} \\
          &= \frac{(\frac{1+\sqrt{5}}{2})^n}{\sqrt{5}}\left(1 + \frac{2}{1+\sqrt{5}}\right) - \frac{(\frac{1-\sqrt{5}}{2})^n}{\sqrt{5}}\left(1 + \frac{2}{1-\sqrt{5}}\right) \\
          &= \frac{(\frac{1+\sqrt{5}}{2})^n}{\sqrt{5}}\left(1 + \frac{2(1 - \sqrt{5})}{-4}\right) - \frac{(\frac{1-\sqrt{5}}{2})^n}{\sqrt{5}}\left(1 + \frac{2(1+\sqrt{5})}{-4}\right) \\
          &= \frac{(\frac{1+\sqrt{5}}{2})^n}{\sqrt{5}}\left(\frac{1 + \sqrt{5}}{2}\right) - \frac{(\frac{1-\sqrt{5}}{2})^n}{\sqrt{5}}\left(\frac{1 - \sqrt{5}}{2}\right) \\
          &= \frac{(\frac{1+\sqrt{5}}{2})^{n+1} - (\frac{1-\sqrt{5}}{2})^{n+1}}{\sqrt{5}},
\end{align*}
as desired.

Finally, since $-1 < ((1-\sqrt{5})/2)^n < 0$ for all $n$, then
\begin{align*}
  \left|F_n - \left(\frac{1+\sqrt{5}}{2}\right)^n\right| = \left|\left(\frac{1-\sqrt{5}}{2}\right)^n\right| < 1,
\end{align*}
which implies that $F_n$ is the closest integer to $((1+\sqrt{5})/2)^n$.

\end{document}
